\documentclass[12pt]{beamer}
\usepackage[utf8]{inputenc}
\usepackage[T1]{fontenc}
%\usepackage[czech]{babel}
\usepackage{lmodern}
\usepackage{graphicx}
\usepackage{montserrat}
\usetheme{Fedora}
%\usecolortheme{seagull}

\title{Co když zmáčknu to červené tlačítko?}
\subtitle{aneb Jak neodpálit jadernou bombu.}
\author{Lukáš Růžička (lruzicka@redhat.com)}
\date{12. prosince 2018, FIT ČVUT Praha}

\begin{document}
\begin{frame}[plain]
\maketitle
\end{frame}

    \section{Proč psát dokumentaci?}
    
    	\begin{frame}{Žádná dokumentace, dobrá dokumentace?}
    	  Dokumentace není potřebná, když je každému hned jasné, co a jak má s danou věcí dělat. Tedy danou věc lze použít
    	  
    	  \begin{itemize}
    	  	\item pouze jedním způsobem
    	  	\item s pevně daným výstupem
    	  	\item a nemožností selhání
    	  \end{itemize}
		\end{frame}
	
		\begin{frame}{Tak raději přece?}
		   O dokumentaci byste měli uvažovat vždy, když:
		   \begin{itemize}
		   	\item existuje více možností, jak danou věc použít
		   	\item věc nabízí více než jednu funkci
		   	\item je možné, že výstup se může lišit v závislosti na podmínkách použití
		   	\item věc nemusí vždy fungovat správně
		   \end{itemize}
		\end{frame}
	
		\begin{frame}{Úkol pro vás}
		Na základě zmíněných kritérií jmenujte jednu věc, pro kterou nepotřebujete dokumentaci.
		\end{frame}
	
		\begin{frame}
			\begin{center}
				\includegraphics[width=10cm]{lahev_bohove.jpg}	\\
				Bohové musí být šílení (1980)
			\end{center}
		
		\end{frame}
	
		\begin{frame}{K přemýšlení}
			\begin{itemize}
			\item Když lidé nebudou o vašem projektu vědět, nebudou ho používat.
			\item Když lidé nebudou umět nainstalovat váš program, nebudou ho používat.
			\item Když lidé nepoznají, jak použít váš kód, nepoužijí ho.
			\item Když váš program nebude mít očekávané výsledky, lidé ho nebudou používat.
			\end{itemize}
		\end{frame}
	
    
    \section{Druhy dokumentace}
    
    	\begin{frame}{Jaký máme výběr?}
    	Mezi dokumentaci můžeme zahrnout několik různých forem:
    	\begin{itemize}
    		\item konceptuální 
    		\item procedurální 
    		\item referenční
    		\item výukovou
    	\end{itemize}
		\end{frame}
	
		\begin{frame}{Konceptuální dokumentace}
		\textbf{Konceptuální} dokumentace se snaží vysvětlit uživatelům danou problematiku (\textbf{koncept}) tak, aby tito pochopili princip fungování a potřebné souvislosti. Například:
		
		\begin{itemize}
			\item Co je počítačová síť a jak se po ní posílají datagramy?
			\item Co jsou optické vlastnosti objektivu a jak ovlivňují výslednou fotografii?
			\item Jak funguje útok \textit{hrubou silou} a proč je důležité volit si komplikovaná přístupová hesla?
			\item Jak působí bakterie zubního plaku a jak zvolit správný kartáček?
		\end{itemize}
		\end{frame}
	
		\begin{frame}{Procedurální dokumentace}
		\textbf{Procedurální} dokumentace poskytuje jasný návod, většinou rozdělený na jednotlivé kroky (\textbf{procedura}), ke splnění uživatelského záměru (\textbf{user case}). Mimo jiné:
	
		\begin{itemize}
			\item Jak nainstalovat aplikaci XY.
			\item Jak napsat a odeslat email pro úplné začátečníky.
			\item Jak vytvořit nového uživatele a přidělit mu patřičná systémová práva.
			\item Jak provést předletovou kontrolu osobního letadla Boeing~747.
		\end{itemize}
		\end{frame}

		\begin{frame}{Referenční dokumentace}
		\textbf{Referenční} dokumentace nabízí ucelený přehled (\textbf{referenci}) vlastností, voleb, nastavení a způsobů použití, aby si uživatel mohl sám objevit vlastní přístup a sestavit si své vlastní postupy:

		\begin{itemize}
			\item manuálové stránky v Linuxu
			\item přehled typů žárovek pro osvětlení vozidla
			\item přehled ingrediencí potřebných pro upečení dortu
			\item přehled voleb příkazu \texttt{dnf} ve Fedoře.
		\end{itemize}
		\end{frame}

  	\begin{frame}{Výuková dokumentace}
		\textbf{Výuková} dokumentace, tzv. \textbf{tutoriály}, je spojení konceptuální a procedurální stránky dokumentace, takže výsledkem není jenom splněný uživatelský záměr, ale také částečné pochopení problematiky a kontextu.
		
		\vspace{5pt}
		
		Je velmi důležité, abychom z konceptuálního hlediska vysvětlili pouze tolik, kolik je nezbytně nutné. Chceme-li pomocí tutoriálů vysvětlit širší problematiku, pak volíme několik na sebe navazujících.	
	\end{frame}

	\begin{frame}{Úkol pro vás}
	Navrhněte dokumentaci pro nějakou věc denní spotřeby. Kterou formu zvolíte? Proč?
	\end{frame}

	\section{Jak psát?}

	\begin{frame}{Správný styl dokumentace}
	V dokumentaci bychom měli zachovávat několik stylistických pravidel:
	\begin{itemize}
		\item jednoznačné výrazy
		\item jednoduché a krátké formulace
		\item neutrální výrazy
		\item genderově neutrální prvky
		\item v případě pochybností je lepší více informací než méně
		\item rozkazovací způsob (procedury)
	\end{itemize}
	\end{frame}

	\begin{frame}{Nesprávný styl dokumentace}
	V dokumentaci bychom se raději měli vyhnout:
	\begin{itemize}
		\item estetickým formám jazyka (metafory, přirovnání, nadsázka)
		\item ironii a sarkasmu
		\item humoru
		\item zlehčování problémů
		\item utěšování uživatelů
	\end{itemize}
	\end{frame}

	\begin{frame}{Porušování pravidel}
	Někdy můžeme uznat za vhodné pravidla porušit a získáme tak jinou formu dokumentace, jež je
	\begin{itemize}
		\item zajímavá
		\item neotřelá
		\item vtipná
		\item parodická
	\end{itemize}
	Je však nutné si uvědomit, kdo budou čtenáři naší dokumentace a jsou-li tito ochotni takový přístup snášet.
	\end{frame}

	\begin{frame}{Úkol pro vás}
			Jakou věc by bylo možné dokumentovat s porušením předchozích pravidel? Jakou věc byste nikdy nechtěli takto dokumentovat?
	\end{frame}

	\section{V čem psát?}

	\begin{frame}{Technické zpracování}
	Dokumentaci můžeme psát ve spoustě různých formátů. Záleží, co přesně od dokumentace očekáváme:
	\begin{itemize}
		\item obtížnosti tvorby dokumentace (WYSIWYG, markdown, markup, \ldots{})
		\item metod zobrazení (web, čtečka, tištěné médium, audiovizuální metody)
		\item možností spolupráce (žádná, cvs, git, Google)
		\item možností publikace (ručně, automaticky (CI))
		\item dostupnosti metadat textu (sémantický markup)
	\end{itemize}

	\end{frame}

	  \begin{frame}{Docs As Code}
S dokumentací zacházíme jako s kódem a používáme:
\begin{enumerate}
	\item popisovací jazyk (Markdown, AsciiDoc, \LaTeX, Docbook)
	\item sdílený repozitář (Gitlab, Github)
	\item správu verzí (Git, SVN)
	\item kontinuální integraci (CI)
	\item automatické publikování (web)
\end{enumerate}
\end{frame}


	\begin{frame}{Formáty}
	Pokud se v souvislosti s dokumentací mluví o formátech, pak máme na mysli v podstatě dva typy a to:
		\begin{itemize}
			\item \textbf{zdrojové} formáty, tedy ty, ze kterých se dokumentace překládá
			\item \textbf{cílové} formáty, tedy ty, do kterých se překládá
		\end{itemize}
	Některé souborové formáty mohou být jak zdrojové, tak cílové (HTML)
	\end{frame}

	\begin{frame}{Stručný přehled zdrojových formátů}
		\begin{itemize}
			\item čistý text
			\item markdown (Markdown, AsciiDoc)
			\item markup (HTML, XML, DocBook, Mallard, reST, \LaTeX)
			\item nativní formáty WYSIWYG aplikací (doc, docx, fm, odt, indd)
		\end{itemize}
	\end{frame}

	\begin{frame}{Stručný přehled cílových formátů}
		\begin{itemize}
			\item HTML (web)
			\item epub, mobi a další (ebooky)
			\item pdf (tiskárna)
		\end{itemize}
	\end{frame}

	\begin{frame}{Nejčastěji používané formáty}
		\begin{itemize}
			\item Markdown (Github)
			\item AsciiDoc (Red Hat)
			\item reST (Python)
			\item Mallard (Gnome)
			\item \TeX, \LaTeX
			\item DocBook (SuSE, IBM)
		\end{itemize}
	\end{frame}

	\begin{frame}{Markdown}
	\textbf{Výhody:}
	\begin{itemize}
		\item jednoduchý
		\item čitelný
	\end{itemize}	
	\textbf{Nevýhody:}
	\begin{itemize}
		\item omezené typografické možnosti
		\item minimální sémantika
		\item nejasné hranice mezi prvky
		\item významy některých značek kolidují s reálnými znaky (*)
	\end{itemize}				
	Markdown je vhodný především pro malé projekty, například \textbf{github.io}.
	\end{frame}

	\begin{frame}{AsciiDoc}
	\textbf{Výhody:}
	\begin{itemize}
		\item jednoduchý a čitelný
		\item částečně programovatelný
		\item lze psát sémantický kód
	\end{itemize}	
	\textbf{Nevýhody:}
	\begin{itemize}
		\item nejasné hranice mezi prvky
		\item vlastní úpravy porušují standard (kompatibilita)
		\item sémantika zvyšuje složitost a omezuje čtivost
		\item kolidující značky
	\end{itemize}				
	AsciiDoc je vhodný pro malé i větší projekty, pokud se spokojíme s omezenými možnostmi.
	\end{frame}
    
	\begin{frame}{reST a Sphinx}
	\textbf{Výhody:}
	\begin{itemize}
		\item jednoduchý a čitelný
		\item programovatelný
		\item sémantický
		\item jasné hranice mezi prvky
	\end{itemize}	
	\textbf{Nevýhody:}
	\begin{itemize}
		\item nutnost přesně dodržovat strukturu kódu
	\end{itemize}				
	reST, a jeho rozšíření Sphinx, jsou velmi vhodné pro veškeré typy projektů. Jsou oblíbené především mezi programátory Pythonu, protože struktura formátu připomíná Python.
	\end{frame}

	\begin{frame}{Mallard}
	\textbf{Výhody:}
	\begin{itemize}
		\item sémantický
		\item jasné hranice mezi prvky
		\item modulární
	\end{itemize}	
	\textbf{Nevýhody:}
	\begin{itemize}
		\item méně přehledný a čtivý
		\item poměrně neznámý
		\item modulární
	\end{itemize}				
	Mallard byl původně určen pro psaní \textbf{Gnome} help. Je navržen pro \textit{topic based authoring} (tématicky zaměřené psaní), takže zdrojový kód se spíše skládá z fragmentovaných prvků než velkých celků.
	\end{frame}

	\begin{frame}{\TeX, \LaTeX}
	\textbf{Výhody:}
	\begin{itemize}
		\item sémantický
		\item jasné hranice mezi prvky
		\item poměrně přehledný a čtivý
		\item dlouhodobě vyvíjený
	\end{itemize}	
	\textbf{Nevýhody:}
	\begin{itemize}
		\item nehezký výstup do HTML
		\item mnoho odnoží s různými výstupy a funkcemi
	\end{itemize}				
	Sázecí systém \TeX{} a sada maker zjednodušujících práci s ním, \LaTeX, byl vyvinut pro sazbu matematických publikací. Nejlepší výsledky poskytuje pro vytváření tištěné dokumentace. 
	\end{frame}

	\begin{frame}{Docbook}
	\textbf{Výhody:}
	\begin{itemize}
		\item sémantický
		\item jasné hranice mezi prvky
		\item dlouhodobě vyvíjený, tedy robustní
		\item systém validace nedovolí chyby
	\end{itemize}	
	\textbf{Nevýhody:}
	\begin{itemize}
		\item nepřehledný
		\item mnoho různých prvků a vztahů mezi nimi
		\item vyžaduje čas na naučení
		\item neodpouští chyby
	\end{itemize}				
	Docbook je pravděpodobně nejkomplexnější volbou pro psaní dokumentace.
	\end{frame}

	\begin{frame}{Výběr vhodného formátu}
	Před výběrem formátu pořádně zamyslet, co od něj přesně chceme, neboť špatný výběr formátu může celou tvorbu dokumentace zkomplikovat.

	Ptejme se na:
	
	\begin{itemize}
		\item formu spolupráce
		\item obtížnost psaní
		\item automatizované testování a publikování
		\item formy výstupu
		\item přenositelnost do jiných formátů
		\item budoucnost projektu
	\end{itemize}
	Obecně platí, že \textbf{čím složitější} formát, tím \textbf{více možností použití}.
	\end{frame}

	\section{Praxe}	

	\begin{frame}{Obvyklý postup při vytváření dokumentace}
	Dokumentace obvykle vzniká následujícím postupem:
			\begin{enumerate}
				\item zjišťování potřeb uživatelů
				\item plánování a alokace zdrojů
				\item stanovení formy spolupráce a formátu
				\item tvorba zdrojových dokumentů
				\item překlad do cílových dokumentů
				\item publikování
				\item vyhledání a opravení chyb
				\item znovu přeložení a publikování (nová subverze)
			\end{enumerate}
	\end{frame}

\begin{frame}{Dokumentace v Red Hatu obsahově}
Red Hat v současné době udržuje mnoho dokumentačních projektů. Organizační struktura dokumentačního oddělení má několik pozic:

\begin{itemize}
	\item obsahový stratég (content strategist)
	\item technický manažer (pillar lead)
	\item manager pro lidské zdroje (people manager)
	\item programový manažer (document program manager)
	\item dokumentátor (technical writer)
\end{itemize}

Každý z nich je zodpovědný za jinou oblast vývoje dokumentace.
\end{frame}


	\begin{frame}{Dokumentace v Red Hatu technicky}
		V současné době se pro většinu dokumentačních projektů v Red Hatu používá 
		\begin{itemize}
			\item modulární přístup
			\item AsciiDoc
		\end{itemize}.
		
	\end{frame}

	\begin{frame}{Pracovní postup}
	Při vytváření dokumentace psané v AsciiDocu se postupuje takto:

	\begin{itemize}
		\item klonování Git repozitáře
		\item ruční úprava dokumentů v textových editorech
		\item sledování změn v paralelní Git větvi (branch)
		\item validace a překlad na lokálním stroji (asciidoctor, ccutil)
		\item \textbf{peer review} $\longrightarrow$ kontrola kvality
		\item \textbf{merge} do hlavní větve
		\item překlad hlavní větve (ccutil $\longrightarrow$ publican) a vystavení na portále
	\end{itemize}
	\end{frame}

	\begin{frame}{Probíhající diskuse o dokumentaci}
	V dokumentačním týmu v minulosti proběhlo několik diskusí a iniciativ o tom, jak dělat věci lépe
	
	\begin{itemize}
		\item spolupráce upstream $\longleftrightarrow$ downstream (upstream first)
		\item konzistentní terminologie (IBM Style Guide)
		\item Docbook versus AsciiDoc (AsciiDoc)
		\item správný postup u peer review (definice postupu)
		\item jednotný způsob zápisu značek u AsciiDocu (doporučený způsob)
		\item sémantický zápis značek v AsciiDocu (zatím se neujalo)
		\item modulární dokumentace (objevují se problémy)
	\end{itemize}
	\end{frame}

	\begin{frame}{Dokumentace pro Fedoru}
\textbf{Fedora} je komunitní distribuce Linuxu, na jejímž vývoji se může podílet kdokoliv. Tak i na její dokumentaci.

	\begin{itemize}
		\item repozitáře na pagure.io (Git)
		\item využívá projekt Antora (antora.org)
		\item vydána na \textbf{docs.fedoraproject.org}
	\end{itemize}

	Přispět můžete po přečtení \\
	\url{https://docs.fedoraproject.org/en-US/fedora-docs/contributing/}

	\end{frame}


	\begin{frame}{Otázky}
		Kdo se moc ptá, moc se doví.
		
		\vspace{10pt}
		
		Takže?
	\end{frame}

	\begin{frame}
	Děkuji za pozornost.
	\end{frame}

\end{document}
