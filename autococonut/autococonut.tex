\documentclass[14pt]{beamer}
\usepackage[utf8]{inputenc}
\usepackage[T1]{fontenc}
\usepackage{lmodern}
\usepackage{graphicx}
\usetheme{Bergen}
\begin{document}
	\author{Lukáš Růžička}
	\title{Pi\~na Colada with AutoCoconut}
	%\subtitle{}
	%\logo{}
	%\institute{}
	%\date{}
	%\subject{}
	%\setbeamercovered{transparent}
	\setbeamertemplate{navigation symbols}{}
	\begin{frame}[plain]
		\maketitle
	\end{frame}
	
	\begin{frame}
		\frametitle{What is AutoCoconut?}
		
		\textbf{AutoCoconut} is a mouse and keyboard events logging and screenshots taking application that helps to easily record a~certain \textbf{workflow}.
	\end{frame} 

\begin{frame}
	\frametitle{How does it work?}
	The application $\ldots$
	\begin{enumerate}
		\item runs in the background
		\item records mouse and key events, takes screenshots
		\item saves all such information in a json file (raw)
		\item interprets the recorded data
		\item wraps them into a file for output (json, adoc, html, openqa)
	\end{enumerate}
\end{frame}

\begin{frame}
	\frametitle{Parts of the application}
	
	\begin{itemize}
		\item \texttt{autococonut.py}		
		\item \texttt{event\_handler.py}
		\item \texttt{screenshot\_grabber.py}
		\item \texttt{interpreter.py}
		\item \texttt{translator.py}
	\end{itemize}
	
\end{frame}

\begin{frame}
	\frametitle{event\_handler.py}
	\begin{itemize}
	\item uses \texttt{pynput} to monitor the input
 	\item records each single event (mouse and key)
	\item classifies keys into groups (alphanumeric, special, modifiers)
	\item matches events and screenshots
	\item creates a json file with ``raw'' events
	\end{itemize}
\end{frame}

\begin{frame}[fragile]
	\frametitle{The `raw' json format}
	\begin{verbatim}
		"1597741812.6215603": {
		    "type": "mouse",
		    "action": "click",
		    "button": "left",
		    "coordinates": [1805,1418],
		    "screens": [
		        "1597741812.6215603.png",
		        "1597741811.6215603.png"]
		},
	\end{verbatim}

\end{frame}

\begin{frame}
	\frametitle{interpreter.py}
	\begin{itemize}
		\item iterates over the `raw' json file
		\item searches for patterns
		\item converts `raw` events into \textbf{super events}
		\item creates a new json with super events
	\end{itemize}
\end{frame}

\begin{frame}
	\frametitle{Interpreter super events}
	\begin{itemize}
		\item click and double click
		\item scroll and drag
		\item typing
		\item special keys
		\item key combinations
	\end{itemize}
\end{frame}

\begin{frame}[fragile]
	\frametitle{The super events json file}
		\begin{verbatim}
	"1597745423.9703004": {
	    "type": "typing",
	    "subtype": "text",
	    "action": "type",
	    "key": "o",
	    "screens": [null,null],
	    "combined": null,
	    "text": "useradd pkolinko",
	    "reason": "special key pressed"
	},
	\end{verbatim}
	
\end{frame}

\begin{frame}
	\frametitle{translator.py}
	\begin{itemize}
	\item iterates over the super events json file
	\item retouches some of the events to comply with the selected output
	\item creates overlays for the screenshots
	\item creates needle json files for OpenQA output
	\end{itemize}	
\end{frame}

\begin{frame}
	\frametitle{autococonut.py}
	\begin{itemize}
	\item runs the application
	\item deals with console arguments
	\item produces output 
	\item uses the \textbf{Killer Ninja} templates\footnote{I know they are called \textit{Jinja}.} to produce selected output.
\end{itemize}	
\end{frame}

\begin{frame}
	\frametitle{Possible output (currently)}
		\begin{itemize}
		\item raw json
		\item super events json
		\item Asciidoc file
		\item HTML file
		\item OpenQA test subroutine
		\item possibly more
		\end{itemize}	
\end{frame}

\begin{frame}
	\frametitle{How to use it?}
	\begin{enumerate}
		\item run the script
		\item prepare your workplace
		\item press the \textbf{stop key} to start listening
		\item do your workflow
		\item press the stop key to stop listening
		\item get the results
	\end{enumerate}
\end{frame}

\begin{frame}
	\frametitle{Useful CLI options 1}
	\begin{description}
		\item[{\color{blue}-\,-stopkey}] a dedicated key to start and stop listening (F10)
		\item[{\color{blue}-\,-offset}] a time offset in seconds to delay or precede the screenshots (1)
		\item[{\color{blue}-\,-caption}] the title of the output document (AutoCoconut -- workflow report)
	\end{description}
\end{frame}

\begin{frame}
	\frametitle{Useful CLI options 2}
	\begin{description}
		\item[{\color{blue}-\,-file}] the name of the output file and directory to save the screenshots (none)
		\item[{\color{blue}-\,-output}] one of the possible outputs (adoc) 
	\end{description}
\end{frame}

\begin{frame}
	\frametitle{Search for Audi A4}
	\begin{itemize}
		\item look for an available used A4
		\item clicking, typing, special keys
		\item asciidoc output
		\item overlain screenshots
	\end{itemize}	
\end{frame}

\begin{frame}
	\frametitle{Add to Gnome Calendar}
	\begin{itemize}
	\item add an event into the Gnome Calendar
	\item clicking, typing, special keys, key combinations
	\item openqa output
	\item openqa needles
	\item successfully tested in OpenQA
\end{itemize}	
\end{frame}

\begin{frame}
	\frametitle{Test basic calculations}
	\begin{itemize}
		\item test basic operations
		\item mouse clicking, special keys
		\item openqa output
		\item openqa needles
		\item successfully tested in OpenQA
	\end{itemize}	
\end{frame}

\begin{frame}
	\frametitle{Q\&A}
	Speak now or forever hold your peace.
\end{frame}

\end{document}