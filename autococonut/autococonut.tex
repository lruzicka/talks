\documentclass[14pt]{beamer}
\usepackage[utf8]{inputenc}
\usepackage[T1]{fontenc}
\usepackage{lmodern}
\usepackage{graphicx}
\usetheme{Bergen}
\begin{document}
	\author{Lukáš Růžička}
	\title{Pi\~na Colada with AutoCoconut}
	%\subtitle{}
	%\logo{}
	%\institute{}
	%\date{}
	%\subject{}
	%\setbeamercovered{transparent}
	%\setbeamertemplate{navigation symbols}{}
	\begin{frame}[plain]
		\maketitle
	\end{frame}
	
	\begin{frame}
		\frametitle{What is AutoCoconut?}
		
		\textbf{AutoCoconut} is a mouse and keyboard events logging and screenshots taking application that helps to easily record a~certain \textbf{workflow}.
	\end{frame} 

\begin{frame}
	\frametitle{How does it work?}
	The application $\ldots$
	\begin{enumerate}
		\item runs in the background
		\item records mouse and key events, takes screenshots
		\item saves all such information in a json file (raw)
		\item interprets the recorded data
		\item wraps them into a file for output (json, adoc, html, openqa)
	\end{enumerate}
\end{frame}

\begin{frame}
	\frametitle{Parts of the application}
	
	\begin{itemize}
		\item \texttt{event\_handler.py}
		\item \texttt{screenshot\_grabber.py}
		\item \texttt{interpreter.py}
		\item \texttt{translator.py}
	\end{itemize}
	
\end{frame}

\begin{frame}
	\frametitle{event\_handler.py}
	\begin{itemize}
	\item uses \texttt{pynput} to monitor the input
 	\item records each single event (mouse and key)
	\item classifies keys into groups (alphanumeric, special, modifiers)
	\item matches events and screenshots
	\item creates a json file with ``raw'' events
	\end{itemize}
\end{frame}

\begin{frame}
	\frametitle{}
\end{frame}

\begin{frame}
	\frametitle{}
\end{frame}
\end{document}