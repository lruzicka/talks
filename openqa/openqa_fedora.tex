\documentclass[12pt,aspectratio=169]{beamer}
\usepackage[utf8]{inputenc}
\usepackage[T1]{fontenc}
\usepackage[czech]{babel}
\usepackage{lmodern}
%\usetheme{Malmoe}
\usetheme{Fedora169}
\usepackage{graphicx}
\usepackage{newverbs}

\newverbcommand{\tc}{\color{blue}}{}

%\usebackgroundtemplate{\includegraphics[width=\paperwidth,height=\paperheight]{./images/background.jpg}}

\begin{document}
	\author{Lukáš Růžička (lruzicka@redhat.com)}
	\title{Validujeme a akceptujeme Fedoru}
	%\subtitle{Máňa říkala, že to není směroplatné}
	\titlegraphic{\includegraphics[height=2cm]{buggie.png}}
	\institute{Fedora QE}
	\date{}
%	\subject{Fedora 29}
	%\setbeamercovered{transparent}
	\setbeamertemplate{navigation symbols}{}

\begin{frame}[plain]
	\maketitle 
\end{frame}

\section{Úvod}

\begin{frame}{Validační a akceptační testy}

Validační testy a akceptační testy ve Fedoře:

\vspace{15pt}

\begin{itemize}
\item probíhají podle \textit{plánu validačních testů} (Release validation test plan)
\item testují \textit{kritéria kvality vydání} (Fedora Release Criteria)
\item z části automatizované (OpenQA, Autocloud)
\item z části manuální (fyzické i virtuální stroje)
\item testují každé sestavení Fedory (nightly validation)
\item a také předfinální sestavení (candidate compose)
\item skládají se z jednotlivých testů (test case)
\item výsledky se zapisují do výsledkových matic (Wikitcms)
\end{itemize}

\vspace{15pt}

{\small \url{http://bit.ly/rvtestplan}}
\end{frame}

\begin{frame}{Oblasti priority}
	
	\begin{itemize}
		\item instalace
		\item základní systém
		\item desktop
		\item server
		\item cloud
	\end{itemize}
\end{frame}

\begin{frame}{Variace}
	
	\begin{itemize}
		\item různá instalační média
		\item architektura počítače
		\item fyzicky nebo virtuálně
		\item stávající nebo předchozí vydání
		\item bios nebo uefi
		\item blokující nebo neblokující
		\item povinné či nepovinné
		\item Losnu nebo Mažňáka
		\item a mnoho dalších
	\end{itemize}

\vspace{10pt}

Vzhledem k mnoha faktorům je nutné testovat \textbf{často a hodně}. 
\end{frame}

\begin{frame}{Milníky}
	\begin{itemize}
	\item základní (alfa)
	\item beta
	\item finální
	\item následující zahrnuje předcházející
\end{itemize}
\end{frame}

\begin{frame}{Kritéria kvality}
	
	\begin{itemize}
		\item co musí vydání (nebo sestavení) splňovat
		\item co se má kdy testovat
		\item jaké jsou priority
		\item jak závažné jsou nalezené chyby
		\item stupně kvality dle fází vývoje -- \textit{základní}, \textit{beta} a \textit{finální} kritéria
	\end{itemize}
	
	\vspace{10pt}
	
	{\small \url{http://bit.ly/rcriteria}}
\end{frame}

\begin{frame}{Testy}
	\begin{itemize}
		\item definice kritéria kvality
		\item popis
		\item nastavení prostředí
		\item provedení testu
		\item očekávané výsledky
	\end{itemize}

	\vspace{10pt}

{\small \url{http://bit.ly/testexample}}
\end{frame}

\begin{frame}{Blokující}
	Je-li test neúspěšný, tak automaticky zdržuje vydání Fedory až do vyřešení (pokud není rozhodnuto jinak)
		
	\vspace{10pt}
	
	\begin{itemize}
		\item Workstation, Server, Cloud, KDE (x86\_64)
		\item Minimal, Xfce (arm)
		\item \textit{basic}, \textit{beta}, \textit{final}
		\item \textit{optional} nejsou blokující
		\item BlockerBug Process
	\end{itemize}
\end{frame}

\begin{frame}{Výsledky}
	\begin{itemize}
		\item zapisují se do výsledkových \textit{matic}
		\item \textit{inprogress}, \textit{pass}, \textit{fail}
		\item bug, poznámka
		\item robotický tester \textbf{coconut}
	\end{itemize}

	\vspace{10pt}

{\small \url{http://bit.ly/examplematrix}}
\end{frame}

\begin{frame}{Připraveno k vydání}
	
	Fedora se považuje z hlediska QA za vhodnou k vydání:
	
	\vspace{10pt}
	
	\begin{itemize}
		\item všechny blokující testy z matic jsou \textbf{pass}
		\item nebyly nalezeny jiné chyby v rozporu s kritérii
	\end{itemize}

	\vspace{10pt}

	Vše ostatní nehraje roli.
\end{frame}

\begin{frame}{Rizika}
	
	\begin{itemize}
		\item jeden neměnný postup (coconut)
		\item omezený hardware (ThinkPad)
		\item virtuální stroje převažují nad fyzickými
		\item pouze minimální rozsah některých testů
		\item nedostatek zdrojů celkově
		\item malá otestovanost neblokujících součástí Fedory
	\end{itemize}
\end{frame}

\begin{frame}{Statistika F31}
	Beta:
	\begin{itemize}
		\item 16 testerů
		\item celkem 562 testů
		\item nejaktivnější tester -- 233 testů
		\item průměrně 35 testů
	\end{itemize}
	Finální:
	\begin{itemize}
		\item 13 testerů
		\item celkem 431 testů
		\item nejaktivnější tester -- 93 testů
		\item průměrně 33 testů
	\end{itemize}
\end{frame}

\begin{frame}{Testujte s námi}
		\begin{itemize}
		\item používejte Fedoru už od beta verze
		\item používejte oblíbený desktop a aplikace
		\item aplikujte na chyby kritéria kvality
		\item testujte i neblokující testy
		\item testujte i to, co už bylo otestováno
		\item nebojte se nás kontakovat s připomínkami a návrhy
		\item stavte se na \textit{\#fedora-qa} (freenode)
		\item hlašte \textit{BlockerBugs} a hlasujte o nich
		\item \textbf{coconut} je sice pracovitý, ale nemá žádnou fantazii
	\end{itemize}
\end{frame}

\begin{frame}{Kde začít?}
	
{\Large \url{bit.ly/rvtestplan}}

\vspace{10pt}
	
({\small \url{https://fedoraproject.org/wiki/QA:Release_validation_test_plan}})

\end{frame}

\begin{frame}{Konec?}

Pokud máte nějaké otázky, \textbf{ptejte se}. Jinak děkuji za pozornost.

\end{frame}


\end{document}