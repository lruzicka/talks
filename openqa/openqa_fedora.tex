\documentclass[12pt,aspectratio=169]{beamer}
\usepackage[utf8]{inputenc}
\usepackage[T1]{fontenc}
\usepackage[czech]{babel}
\usepackage{lmodern}
%\usetheme{Malmoe}
\usetheme{Fedora169}
\usepackage{graphicx}
\usepackage{newverbs}

\newverbcommand{\tc}{\color{blue}}{}

%\usebackgroundtemplate{\includegraphics[width=\paperwidth,height=\paperheight]{./images/background.jpg}}

\begin{document}
	\author{Lukáš Růžička (lruzicka@redhat.com)}
	\title{openQA for testing Fedora on Fedora}
	%\subtitle{Máňa říkala, že to není směroplatné}
	\titlegraphic{\includegraphics[height=2cm]{buggie.png}}
	\institute{Fedora Quality Engineering}
	\date{}
%	\subject{Fedora 29}
	%\setbeamercovered{transparent}
	\setbeamertemplate{navigation symbols}{}

\begin{frame}[plain]
	\maketitle 
\end{frame}

\section{What is openQA}

\begin{frame}{What is openQA?}

\textbf{openQA} is an automated test tool, developed by SuSE, that allows to test various features of operating systems using the \textit{hands-on} approach:

\vspace{15pt}

\begin{itemize}
\item it runs the operating system
\item it provides an interface (console, GUI)
\item it takes input, passes it to the machine and triggers actions
\item it evaluates the reactions
\end{itemize}

\vspace{15pt}

It behaves similarly to a \textit{human} tester.
\end{frame}

\begin{frame}{Based on}
	
	\begin{description}
		\item[qemu] runs the machines
		\item[VNC] provides the screens
		\item[perl scripts] define actions and evaluations
	\end{description}
\end{frame}

\begin{frame}{openQA architecture}
	\begin{description}
		\item[Controller]
			\begin{itemize}
				\item Web UI (control and visualization)
				\item job handling (scheduling, starting, stopping)
				\item live viewing and interaction (monitoring and development)
				\item results
				\item database (postgresql)
			\end{itemize}
		\item[Worker]
			\begin{itemize}
				\item assets (iso or qcow2)
				\item tests and needles 
				\item syncing and caching 
				\item video recording
			\end{itemize}
	\end{description}
\end{frame}

\begin{frame}{openQA architecture diagram}
TBD: Diagram of the architecture
\end{frame}

\section{Concepts}

\begin{frame}{Assets}


	
	\begin{itemize}
		\item co musí vydání (nebo sestavení) splňovat
		\item co se má kdy testovat
		\item jaké jsou priority
		\item jak závažné jsou nalezené chyby
		\item stupně kvality dle fází vývoje -- \textit{základní}, \textit{beta} a \textit{finální} kritéria
	\end{itemize}
	
	\vspace{10pt}
	
	{\small \url{http://bit.ly/rcriteria}}
\end{frame}

\begin{frame}{Testy}
	\begin{itemize}
		\item definice kritéria kvality
		\item popis
		\item nastavení prostředí
		\item provedení testu
		\item očekávané výsledky
	\end{itemize}

	\vspace{10pt}

{\small \url{http://bit.ly/testexample}}
\end{frame}

\begin{frame}{Blokující}
	Je-li test neúspěšný, tak automaticky zdržuje vydání Fedory až do vyřešení (pokud není rozhodnuto jinak)
		
	\vspace{10pt}
	
	\begin{itemize}
		\item Workstation, Server, Cloud, KDE (x86\_64)
		\item Minimal, Xfce (arm)
		\item \textit{basic}, \textit{beta}, \textit{final}
		\item \textit{optional} nejsou blokující
		\item BlockerBug Process
	\end{itemize}
\end{frame}

\begin{frame}{Výsledky}
	\begin{itemize}
		\item zapisují se do výsledkových \textit{matic}
		\item \textit{inprogress}, \textit{pass}, \textit{fail}
		\item bug, poznámka
		\item robotický tester \textbf{coconut}
	\end{itemize}

	\vspace{10pt}

{\small \url{http://bit.ly/examplematrix}}
\end{frame}

\begin{frame}{Připraveno k vydání}
	
	Fedora se považuje z hlediska QA za vhodnou k vydání:
	
	\vspace{10pt}
	
	\begin{itemize}
		\item všechny blokující testy z matic jsou \textbf{pass}
		\item nebyly nalezeny jiné chyby v rozporu s kritérii
	\end{itemize}

	\vspace{10pt}

	Vše ostatní nehraje roli.
\end{frame}

\begin{frame}{Rizika}
	
	\begin{itemize}
		\item jeden neměnný postup (coconut)
		\item omezený hardware (ThinkPad)
		\item virtuální stroje převažují nad fyzickými
		\item pouze minimální rozsah některých testů
		\item nedostatek zdrojů celkově
		\item malá otestovanost neblokujících součástí Fedory
	\end{itemize}
\end{frame}

\begin{frame}{Statistika F31}
	Beta:
	\begin{itemize}
		\item 16 testerů
		\item celkem 562 testů
		\item nejaktivnější tester -- 233 testů
		\item průměrně 35 testů
	\end{itemize}
	Finální:
	\begin{itemize}
		\item 13 testerů
		\item celkem 431 testů
		\item nejaktivnější tester -- 93 testů
		\item průměrně 33 testů
	\end{itemize}
\end{frame}

\begin{frame}{Testujte s námi}
		\begin{itemize}
		\item používejte Fedoru už od beta verze
		\item používejte oblíbený desktop a aplikace
		\item aplikujte na chyby kritéria kvality
		\item testujte i neblokující testy
		\item testujte i to, co už bylo otestováno
		\item nebojte se nás kontakovat s připomínkami a návrhy
		\item stavte se na \textit{\#fedora-qa} (freenode)
		\item hlašte \textit{BlockerBugs} a hlasujte o nich
		\item \textbf{coconut} je sice pracovitý, ale nemá žádnou fantazii
	\end{itemize}
\end{frame}

\begin{frame}{Kde začít?}
	
{\Large \url{bit.ly/rvtestplan}}

\vspace{10pt}
	
({\small \url{https://fedoraproject.org/wiki/QA:Release_validation_test_plan}})

\end{frame}

\begin{frame}{Konec?}

Pokud máte nějaké otázky, \textbf{ptejte se}. Jinak děkuji za pozornost.

\end{frame}


\end{document}