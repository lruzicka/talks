\documentclass[12pt,a4paper]{article}
\usepackage[utf8]{inputenc}
\usepackage[czech]{babel}

\title{Vektorová grafika v Inkscape}
\author{Mgr. Lukáš Růžička}
\date{18. dubna 2019}

\begin{document}
\maketitle

\section{Co je vektorová grafika?}
	\begin{itemize}
		\item Co to je?
	 	\item Na co se to nejvíce hodí (diagramy, šipky, čáry, obrazce)
	 	\item Síla vektorové grafiky -- škálovatelnost, jednoduchost
	\end{itemize}

\section{Nastavení dokumentu Inkscape}
\begin{itemize}
	\item Nastavení velikosti, orientace, metriky.
	\item Mřížka a vodítka.
\end{itemize}

\section{Základní objekty}
	\begin{itemize}
		\item Čtverec (F4), kruh (F5), hvězda (*), kreslení od ruky (F6), Beziérova křivka (Shift-F6), elipsa (F9), text (F8), kaligrafie (Shift-F6), spray, 
		\item Manipulace s objektem -- posouvání, zvětšování, rotace, naklánění, zrcadlení, převracení
		\item Přesné rozměry, upozornit na možnost vypnout škálování orámování.
		\item Práce s výplní a rámováním (paleta, styl čáry, výplň, rámování, přechod)
		\item Seskupování objektů, zarovnávání objektů
		\item Kótování -- Rozšíření, vizualizace křivky, rozměry
	\end{itemize}	

\section{Práce s křivkou}
 \begin{itemize}
		\item Beziérova křivka -- čára, křivka, šipky
		\item Uzly na křivce -- přidání a odebrání uzlu
		\item Průběhy uzlu (roh, zaoblení, souměrnost)
		\item Spojování a rozpojování uzlů
		\item Sjednocení, rozdíl, průnik u objektů (pozor na padání programu)
	\end{itemize}
	
\section{Práce s textem}
	
\begin{itemize}
		\item Psaní textu
		\item Text v textovém rámci
		\item Prostrkání, mezislovní mezery, kerning, vertikální posun
		\item Text na křivce
\end{itemize}	

\section{Práce s vrstvami}
\begin{itemize}
	\item Popředí a pozadí v rámci jedné vrstvy a překrývání objektů.
	\item Co jsou to vrstvy?
	\item Vrstva -- přidání, pojmenování, odebrání, uzamčení
\end{itemize}	

\section{Ukládání a export}
\begin{itemize}
	\item Ukládání do SVG (nativní formát)
	\item Tiskové formáty - EPS a PDF.
	\item Export do PNG - rozlišení
\end{itemize}

\section{Didaktické využití}
	
\begin{itemize}
	\item Učebních materiálů
	\item Infografiky
	\item Obrázky do prezentací
	\item Grafické prvky na webové stránky
\end{itemize}	

\section{Více informací}
\begin{itemize}
	\item Stránky Inkscape -- www.inkscape.org
	\item OpenClipArt -- obrázky zdarma pro veškeré použití
\end{itemize}

\end{document}
