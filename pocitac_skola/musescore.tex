\documentclass[12pt,a4paper]{article}
\usepackage[utf8]{inputenc}
\usepackage[czech]{babel}

\title{ Píšeme hudbu v MuseScore (plán workshopu)}
\author{Mgr. Lukáš Růžička}
\date{17. dubna 2019}

\begin{document}
\maketitle

\section{Začínáme psát (základy notace)}
	\begin{itemize}
		\item  Vytvoření nového dokumentu (Ctrl-N) a vyplnění údajů o skladbě (nepovinné)
	 	\item Vkládání předznamenání, taktu, tempa, a podobně 
	 	\item Vkládání not (klávesnice, myš, klaviatura (P)), také zmínit další volby (N)
	 	\item Ukázat vkládání not jednotlivými způsoby.
	\end{itemize}

\section{Mírně pokročilá notace}
	\begin{itemize}
		\item Triky a tipy při vkládání - rozdělení pomlky na menší, sloučení pomlky
		\item Ukázat možnost změny výšky noty pomocí šipek nebo myši.
		\item Ukázat možnost změny délky noty pomocí čísel nebo kliknutí na symbol. Varovat o porušení notace.
		\item Přidávání not do akordů (Shift - písmeno)
		\item Ukázat možnost Kopírování a vložení, upozornit na nutnost přesného umístění
		\item Ukázat Palety nástrojů a projít možnosti
		\item Přidávání symbolů z palety a jejich mazání
		\item Přidávání textu (Kliknout na notu a CTRL-L)
		\item Vkládání dalších prvků (Menu Add)
	\end{itemize}
	
\section{Úprava partitury a nástrojů}
	
\begin{itemize}
		\item  Nastavení jednotlivých nástrojů v partituře - přidání, odebrání, typ notace. (Edit - Instruments (I))
		\item Výměna nástroje - přidat nástroj, zkopírovat do něj noty z jiného, a ten pak odebrat
		\item Vysvětlit co to je Concert Pitch (znějící výška), nástroje s jiným předznamenáním zobrazí jako normální nástroje
		\item Vyvádění hlasů pro tisk partů
\end{itemize}	

\section{Přehrávání, změna zvuků nástrojů a export}
	
\begin{itemize}
	\item Přehrávání (F11) - region, opakování regionu, metronom, úvodní odklepání
	\item Mixer a základní efekty.
	\item Nastavení Syntezátoru (soubory sf2), upozornit na možnost změny ladění (Tuning)
	\item Ukládání a export (nativně, pdf, midi, wav, ogg, flac)
\end{itemize}	

\section{Didaktické využití}
	
\begin{itemize}
	\item Možnost pracovat s MuseScore pomocí interaktivní tabule, nebo na jednotlivých počítačích (skupinová či individuální práce), nebo doma.
	\item Možnost aktivně a kreativně tvořit (podobně jako ve výtvarné výchově), přičemž výsledkem je konkrétní hmatatelný produkt
	\item Možnost vizualizovat noty a spojit je se zvukem, výborné u stupnic a akordů.
	\item Práce s melodickou linkou a doprovodem. \enlargethispage{-60pt}
	\item Možnost aktivního procvičování (nebo zkoušení)
	\begin{itemize}
		\item Napiš stupnici.
		\item Vytvoř akord.
		\item Doplň chybějící noty do známé melodie.
		\item Doplň chybějící takty do známé melodie.
		\item Doplň správné předznamenání, aby notový zápis (bez změny not) odpovídal známé melodii.
		\item Vytvoř rytmický doprovod (bicí linku) pro různé hudební styly: polka, valčík, rocková balada, samba, \ldots{}
		\item Zapiš do not jednoduchou píseň. \\
		\ldots{} a mnoho dalších možností
	\end{itemize}

\end{itemize}	

\section{Více informací}
\begin{itemize}
	\item Stránky MuseScore -- www.musescore.org.
	\item Příručka MuseScore zde: https://musescore.org/cs/handbook
	\item Hudební notace ke stažení zdarma: https://musescore.com/dashboard
\end{itemize}

\end{document}