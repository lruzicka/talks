\documentclass[12pt]{beamer}
\usepackage[utf8]{inputenc}
\usepackage[T1]{fontenc}
\usepackage{lmodern}
%\usetheme{Malmoe}
\usetheme{Ilmenau}
\usecolortheme{orchid}
\usepackage{graphicx}
\usepackage{newverbs}

\newverbcommand{\tc}{\color{blue}}{}

%\usebackgroundtemplate{\includegraphics[width=\paperwidth,height=\paperheight]{./images/background.jpg}}

\begin{document}
	\author{Lukáš Růžička (lruzicka@redhat.com)}
	\title{Bugzilla Kills Godzilla}
	\subtitle{or How to report bugs the useful way}
	\titlegraphic{\includegraphics[height=2cm]{buggie.png}}
	\institute{Fedora QE}
	\date{}
%	\subject{Fedora 29}
	%\setbeamercovered{transparent}
	%\setbeamertemplate{navigation symbols}{}

\begin{frame}[plain]
	\maketitle 
\end{frame}

\begin{frame}{What is Bugzilla?}
Bugzilla is a bug-tracking (issue-tracking) system. Among the most important features are:

\begin{itemize}
\item open source
\item powerful bug tracking
\item highly configurable
\item history aware
\item robust and stable
\item secure
\item various interfaces (configurable and localisable)
\end{itemize}

More at {\color{blue}\url{www.bugzilla.org}}.
\end{frame}

\begin{frame}{Red Hat Bugzilla}
Bugzilla is the bug-tracking system used at Red Hat. It is available to everyone (customers, collaborators, developers, QAs and more) interested in:

\begin{itemize}
	\item Red Hat products
	\item JBoss products
	\item Fedora products
	\item Community projects
	\item Internal products
\end{itemize}

Red Hat Bugzilla lives at {\color{red}\url{bugzilla.redhat.com}}.
\end{frame}

\begin{frame}{Why bug tracking?}

By tracking you bugs, you will

\begin{itemize}
	\item receive information about a problem
	\item organize your time, plan, and estimate
	\item cooperate with others
	\item share knowledge and ideas
	\item see the progress
	\item find a solution.
\end{itemize}

\end{frame}


\end{document}